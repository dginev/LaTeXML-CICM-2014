\documentclass{llncs}
%\usepackage[citestyle=alphabetic,style=alphabetic,urldate=long]{biblatex}
\usepackage{xspace}
\usepackage{latexml}
%\usepackage[show]{ed}
\def\ebook{\mbox{E-book}\xspace}
\def\ebooks{\mbox{E-books}\xspace}
\def\SVG{\texttt{SVG}\xspace}

\title{\ebooks and Graphics with \LaTeXML}
\author{Deyan Ginev\inst{1} \and Bruce~R.~Miller\inst{2} \and Silviu Oprea\inst{3}}
\institute{Computer Science, Jacobs University Bremen, Germany
 \and National Institute of Standards and Technology, Gaithersburg, MD, USA
 \and University of Oxford, Oxford, UK.}
\date{\today}

\begin{document}
\maketitle
\begin{abstract}
Marked by the highlights of native generation of EPUB \ebooks and Tikz support for creating SVG images, we present an annual report of {\LaTeXML} development in 2013. {\LaTeXML} provides a reimplementation of the TeX parser, geared towards preserving macro semantics; it supports an array of output formats, notably HTML5, EPUB, XHTML and its own TeX-near XML. 

Other highlights include enhancing performance when used inside high-throughput build systems, via incorporating a native ZIP archive workflow, as well as a simplified installation procedure that now allows to deploy LaTeXML as a cloud service. To this end, we also introduce an official plugin-based scheme for publishing new features that go beyond the core scope of LaTeXML, such as web services or unconventional post-processors.

The software suite has now migrated to GitHub and we welcome forks and patches from the wider FLOSS community.
\end{abstract}

\section{Introduction}
Another busy year of \LaTeXML\ development has gone by;
while we've not completely accomplished all the tasks we'd hoped for,
we've finished others including some we hadn't originally planned.

% OLD:
% \LaTeXML\ \cite{Miller:latexml:online} is a \TeX\ to XML converter, bringing the well-known authoring syntax
% of \TeX\ and \LaTeX\ to the world of XML. Not a new face in the MKM crowd, \LaTeXML\
% has been adopted in a wide range of MKM applications. Originally designed to
% support the development of NIST's Digital Library of Mathematical Functions (DLMF),
% it is now employed in publishing frameworks, authoring suites and for the preparation
%  of a number of large-scale \TeX\ corpora.

\begin{itemize}
\item finished move to GitHub, fork us and send patches!
\item Project is alive and stronger than ever (do "gitstats /path/to/git-checkout-of/LaTeXML html; firefox html/index.html"). We have had a steady increase in commits - 200 in 2010, 400 in 2011, 600 in 2012 and 800 in 2013.
\item We have a brand new CPAN-friendly, plugin-based contribution model for fancy features
\item added support for EPUB (discussion on mobi?)
\item added support for Tikz (dsicsussion on XYPic and other formats?)
\item native ZIP conversion workflow, with arXiv's algorithm for determining main TeX files.
\item ? one-pass conversion to HTML (with Bibliography)
\item ? Additional commitment to quality control in development (perltidy, perlcritic)
\item cloud-friendly - now possible to install locally with cpanm on cloud platform such as Heroku (I will try that today!) 
\end{itemize}

\section{Reorganization}\label{reorganization}
We have reorganized both our code development and our code base.
In the first sense, we have moved our repository to GitHub,
see \url{https://github.com/brucemiller/LaTeXML}
where you can more conveniently browse our code, or check out the latest version.
We have also ported our Trac ticket database to GitHub's Issues,
so you can also report bug and request features from the same place.

Along with the move to github came more opportunities to share
code and development which called for clearer code standards.
We have made a commitment to code quality and formatting by
adopting \texttt{perltidy} and \texttt{perlcritic} policies
which are automatically enforced by \texttt{git} mechanisms.

In the second sense, we have reorganized the code to more clearly
separate the code related to the separate phases of processing,
but allow better connection and code sharing for when more
complex processing is called for, such as carrying a single \TeX\ source
file through the full processing to \HTML, or even ePub (see \ref{ebooks}).
In particular, it provides better support for the Daemon mode of
usage, with a separate web-service module available.

This reorganization positions us to develop an 
plugin architecture that will allow modular extensions
that cover both new \LaTeX\ styles and bindings, but also
include enhanced postprocessing for more sophisticated applications
such as s\TeX.

And some tentative plugins are already available?

\section{\ebooks}\label{ebooks}
The newest version of ePub, version 3, incorporates \MathML\
to represent mathematics.  \LaTeXML\ is already generating \HTML,
with embedded \MathML, and allows that output to be split into
multiple pages as specified by the user. Additionally, it
provides for zipping up the resulting directory into a \texttt{zip}
archive.  With appropriate rearrangment of the peices,
and the addition of a Manifest of the correct structure,
we have the basics needed to generate ePub documents.

That sounds too easy, doesn't it?

\section{Graphics}\label{graphics}
Before we turn our attention to graphics,
a brief digression may be in order.
There are two main approaches used to generate
\HTML\ from \TeX. The first approach,
exemplified by \texttt{tex4ht}, uses the \TeX\ engine
itself by redefining certain commands to drop
\verb|\special| data into the \texttt{dvi} output file.
Instead of \texttt{dvips}, a special \texttt{dvi} processor then deciphers
the \texttt{dvi} and \verb|\specials|
to infer and construct the appropriate \HTML.
In the second approach, used by \LaTeXML, a program
is developed which emulates \TeX\ for the most part
but interprets some macros (called ``Constructors'' in \LaTeXML)
specially, so that it produces \XML\ directly.

The first approach has the advantage of at least
allowing the processing of arbitrary \TeX\ and \LaTeX\ packages,
whether or not the resulting \HTML\ ...
The challenge is in the \TeX\ programming necessary to
insert the \verb|\specials|, generating valid \HTML,
and in whether sufficient semantic structure can be
recovered from the \texttt{dvi}.

The second approach has the advantage of having more
direct control of the generated output.
It is easier, though not trivial, to extend to new
\XML\ structures.

The challenge is to emulate \TeX\ sufficiently
to process complex packages.
Furthermore, \LaTeXML:
uses an intermediate \XML\ format which preserves
the semantic structure;
is fundamentally XML aware, so it produces valid \XML;
A feature of \LaTeXML\ bindings is that macro
control sequences can be defined to be ``Constructors''
which directly construct the \XML\ representation
of their content.

In either approach, \LaTeX\ packages that define
macros with semantic intent must be dealt with
individually or else the semantics will be lost.

Within that context, we were skeptical when Michael Kohlhase
initially posed the proposition: Was \LaTeXML's engine good
enough to implementing the \texttt{tikz} package?
Presumably any semantics implied by tikz markup isn't so critical?
The package is so large and complex, not to mention
its development so fast-moving, that creating \LaTeXML-specific
bindings for all its commands impractical.  However,
it is designed with a relatively small driver layer
in mind, and even has a \texttt{tex4ht} driver for producing SVG!

The main tasks, then, were to implement \LaTeXML\ bindings
for that driver and improve \LaTeXML's engine to cope
with the sophisticated \TeX\ macro usage in the higher
layers of \texttt{pgf} and \texttt{tikz}.

Ultimately, we succeeded beyond our expectations.
Although the results are not perfect,
\LaTeXML\ now successfully processes 3/4 of the
first page of \texttt{tikz} examples on the
\url{http:texamples.net} site, generating valid
\HTML5, with text and \MathML\ combined.
In contrast, \texttt{tex4ht} succeeds on slightly more
than half the examples, often producing invalid markup,
and doesn't support \MathML\ embedded in the \SVG.
It must be admitted, however, that \LaTeXML\ is \emph{very}
slow at processing \texttt{tikz} markup!

In the process, we have further improved the
fidelity of the \TeX\ emulation, introduced
a (currently very rudimentary) mechanism for estimating
the size of displayed objects and exercised the
integration of both \MathML\ and \SVG\ into \HTML.
These improvements are beneficial even outside the graphics

Areas needing further work are \texttt{tikz}' matrix
structure which currently clashes with \LaTeXML's handling
of alignments; inaccuracies of \LaTeXML's sizing of objects;
and, of course, examples involving other exotic packages
not yet known to \LaTeXML.

We are hoping to leverage this experience and apply
it to supporting the \texttt{xy} packages, another
popular and powerful system.  It seems to have a less
well-defined driver layer and we are in the early stages of
discovering a small set of macros that could serve that
function.  We've had some preliminary success, however.

We already have minimal support, for the \texttt{pstricks}
package, but with its Postscript oriented design,
it is more time consuming to develop further bindings.

\section{Outlook}
% OLD:
% Although development was never stagnated, an official release is long overdue;
% a \LaTeXML\ 0.8 release is planned for early 2014.
% It will incorporate the enhancements presented here:
% support for several \LaTeX\ graphics packages, such as Tikz and Xypic;
% an overhauled XSLT and CSS styling framework;
% and a merge of daemonized processing to the master branch.

Daemon is merged, 

\bibliographystyle{alpha}
\bibliography{kbib/kwarc}
%\printbibliography

\end{document}
