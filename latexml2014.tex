\documentclass{llncs}
%\usepackage[citestyle=alphabetic,style=alphabetic,urldate=long]{biblatex}
\usepackage{xspace}
\usepackage{latexml}
%\usepackage[show]{ed}
\def\ebook{\mbox{E-book}\xspace}
\def\ebooks{\mbox{E-books}\xspace}
\title{\ebooks and Graphics with \LaTeXML}
\author{Deyan Ginev\inst{1} \and Bruce~R.~Miller\inst{2}}
\institute{Computer Science, Jacobs University Bremen, Germany
 \and National Institute of Standards and Technology, Gaithersburg, MD, USA}
\date{\today}

\begin{document}
\maketitle
\begin{abstract}
Marked by the highlights of native generation of EPUB \ebooks and Tikz support for creating SVG images, we present an annual report of {\LaTeXML} development in 2013. {\LaTeXML} provides a reimplementation of the TeX parser, geared towards preserving macro semantics; it supports an array of output formats, notably HTML5, EPUB, XHTML and its own TeX-near XML. 

Other highlights include enhancing performance when used inside high-throughput build systems, via incorporating a native ZIP archive workflow, as well as a simplified installation procedure that now allows to deploy LaTeXML as a cloud service. To this end, we also introduce an official plugin-based scheme for publishing new features that go beyond the core scope of LaTeXML, such as web services or unconventional post-processors.

The software suite has now migrated to GitHub and we welcome forks and patches from the wider FLOSS community.
\end{abstract}

\section{Introduction}

% OLD:
% \LaTeXML\ \cite{Miller:latexml:online} is a \TeX\ to XML converter, bringing the well-known authoring syntax
% of \TeX\ and \LaTeX\ to the world of XML. Not a new face in the MKM crowd, \LaTeXML\
% has been adopted in a wide range of MKM applications. Originally designed to
% support the development of NIST's Digital Library of Mathematical Functions (DLMF),
% it is now employed in publishing frameworks, authoring suites and for the preparation
%  of a number of large-scale \TeX\ corpora.

\section{Outlook}
% OLD:
% Although development was never stagnated, an official release is long overdue;
% a \LaTeXML\ 0.8 release is planned for early 2014.
% It will incorporate the enhancements presented here:
% support for several \LaTeX\ graphics packages, such as Tikz and Xypic;
% an overhauled XSLT and CSS styling framework;
% and a merge of daemonized processing to the master branch.

\bibliographystyle{alpha}
\bibliography{kbib/kwarc}
%\printbibliography

\end{document}
