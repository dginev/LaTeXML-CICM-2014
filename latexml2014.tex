\documentclass{llncs}
\usepackage[citestyle=alphabetic,style=alphabetic,urldate=long]{biblatex}
\bibliography{kbib/kwarc}
\usepackage{latexml}
%\usepackage[show]{ed}
\title{E-books and Graphics with \LaTeXML}
\author{Deyan Ginev\inst{1} \and Bruce~R.~Miller\inst{2}}
\institute{Computer Science, Jacobs University Bremen, Germany
 \and National Institute of Standards and Technology, Gaithersburg, MD, USA}
\date{\today}

\begin{document}
\maketitle
\begin{abstract}
Abstract goes here.
\end{abstract}

\section{Introduction}

\LaTeXML\ \cite{Miller:latexml:online} is a \TeX\ to XML converter, bringing the well-known authoring syntax
of \TeX\ and \LaTeX\ to the world of XML. Not a new face in the MKM crowd, \LaTeXML\
has been adopted in a wide range of MKM applications. Originally designed to
support the development of NIST's Digital Library of Mathematical Functions (DLMF),
it is now employed in publishing frameworks, authoring suites and for the preparation
 of a number of large-scale \TeX\ corpora.


\section{Outlook}
Although development was never stagnated, an official release is long overdue;
a \LaTeXML\ 0.8 release is planned for early 2014.
It will incorporate the enhancements presented here:
support for several \LaTeX\ graphics packages, such as Tikz and Xypic;
an overhauled XSLT and CSS styling framework;
and a merge of daemonized processing to the master branch.

\printbibliography

\end{document}
